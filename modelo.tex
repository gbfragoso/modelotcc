\NeedsTeXFormat{LaTeX2e}
%-----------------------------------------------------------
\documentclass[a4paper,12pt]{monografia}
\usepackage[brazil]{babel}
\usepackage[utf8]{inputenc}
\usepackage[T1]{fontenc}
\usepackage{indentfirst}
\usepackage{hyphenat}
\usepackage{csquotes}
\usepackage{courier}
\usepackage{graphicx}
\usepackage{listings}
\usepackage{color}
\usepackage[
backend=biber,
style=alphabetic,
citestyle=authoryear
]{biblatex}

% Estilo para trechos de códigos
\definecolor{codebg}{RGB}{240,240,240}
\lstdefinestyle{codigo}{
    backgroundcolor=\color{codebg},
    keywordstyle=\color{blue},
    basicstyle=\footnotesize,
    breakatwhitespace=false,         
    breaklines=true,                 
    captionpos=b,                    
    keepspaces=true,                 
    numbers=left,                    
    numbersep=5pt,                  
    showspaces=false,                
    showstringspaces=false,
    showtabs=false,                  
    tabsize=2
}
 
\lstset{style=codigo}

\addbibresource{biblio.bib}
\renewcommand*{\nameyeardelim}{\addcomma\space}

% Comando para imagens onde se deve destacar a fonte
\newcommand*{\captionsource}[2]{%
  \caption[{#1}]{%
    #1%
    \\\hspace{\linewidth}%
    \textbf{Fonte:} #2%
  }%
}

%\usepackage[hidelinks]{hyperref}
%-----------------------------------------------------------
\begin{document}
%-----------Título e Dados do Autor---------------------------
\titulo{COLOQUE SEU TÍTULO AQUI}
%\subtitulo{Subtítulo} % opcional
\autor{SEU NOME AQUI} \nome{Primeironome} \ultimonome{Ultimonome}
%
%-----------Informe o Curso e Grau----------------------------
\bacharelado \curso{Bacharelado em Ciência da Computação} \ano{\textbf{2017}}
\data{16/02/2017} % data da aprovação
\cidade{Cidade}
\estado{Estado}
%
%-----------Informações sobre a Instituição-------------------
\instituicao{UNIVERSIDADE ESTADUAL DE SANTA CRUZ} \sigla{UESC}
\unidadeacademica{DEPARTAMENTO DE CIÊNCIAS EXATAS E TECNOLÓGICAS}

%-----------Informações obtidas na Biblioteca-----------------
%\CDU{XXX.XX} \areas{1.???  2.???}
%\npaginas{xx}  % total de páginas do trabalho
%------Nomes do Orientador, 1o. Examinador e 2o. Examinador-
\orientador{Orientador}
%
%\coorientador{Nome do Co-orientador} % opcional
%
\examinadorum{Examinador 1}
%
\examinadordois{Examinador 2}
%
%\examinadortres{Nome do Examinador 3}
%
%\examinadorquatro{Nome do Examinador 4}
%--------- Títulos do Orientador 1o. e 2o. Examinadores ----
\ttorientador{Doutor}
%
%\ttcoorientador{Título do Co-orientador} % se digitado \coorientador
%
\ttexaminadorum{Doutor}
%
\ttexaminadordois{Doutor}
%
%\ttexaminadortres{Título do Examinador 3}
%
%\ttexaminadorquatro{Título do Examinador 4}
%
%--------- Funções do 1o. e 2o. Examinadores ---------------
\funcaoexaminadorum{Professor da Disciplina}
\funcaoexaminadordois{Professor do Curso de Ciência da Computação}
%
\maketitle

%-----------Dedicatória (Opcional)--------------------------
\begin{dedicatoria}
Dedico à minha família, por todo o apoio e confiança.
\end{dedicatoria}

\maketitle

%-----------Agradecimentos (Opcional)-----------------------
%\agradecimento{AGRADECIMENTOS}
% Digite seu texto de agradecimento aqui
%\newpage
% ou inclua um arquivo .tex como mostrado abaixo
\include{agradecimentos}
\newpage

%-----------Epígrafe (Opcional)-----------------------------
%\begin{epigrafe}
%``Lembra que o sono é sagrado e alimenta de horizontes o tempo acordado de viver''.\\
%\hfill Beto Guedes (Amor de Índio)
%\end{epigrafe}

%-----------Digite aqui o seu resumo em Português-----------
%resumo{Resumo}
%\noindent Palavras-chave:
\resumo{RESUMO}
Morbi efficitur molestie pellentesque. Fusce tincidunt vitae dolor ac ornare. Mauris nibh mi, condimentum nec ex a, semper posuere augue. Ut sagittis condimentum lacus, et lacinia sem ornare nec. Praesent cursus sagittis lacus ut iaculis. Nunc faucibus, elit ac imperdiet malesuada, velit est faucibus diam, vitae ullamcorper sapien augue a nisi. Morbi consectetur pulvinar felis vel feugiat. Phasellus tempor magna eget purus placerat luctus. Vestibulum bibendum dapibus arcu in semper. Phasellus vel porta mauris. Ut nec mauris vel ante auctor vulputate a sed nisi. Nam ullamcorper purus vel dolor interdum efficitur. Aenean rhoncus mollis porta. Vivamus est urna, finibus vel leo at, porta tempus sem.   

\noindent Palavras-chave: 

%
%-----------Digite aqui o seu resumo em Inglês--------------
%\resumo{Abstract} 
%\noindent Keywords:
\resumo{ABSTRACT}
Curabitur malesuada ante lorem, a auctor urna euismod et. Nam viverra, dolor eu feugiat euismod, justo velit tincidunt purus, faucibus interdum mauris metus in turpis. Maecenas hendrerit, felis quis condimentum convallis, metus turpis porttitor ex, non iaculis nisi ex id ligula. Vivamus sed consectetur felis. Maecenas non ligula eu nulla iaculis dictum. Phasellus accumsan tempus purus et consectetur. Praesent dapibus, arcu ut porta dictum, velit lacus ultricies nisl, vitae congue purus mi id ipsum. Pellentesque ac tempus enim, at egestas nulla. Quisque vitae ultrices odio. Lorem ipsum dolor sit amet, consectetur adipiscing elit. Sed vitae purus ultricies, maximus magna a, aliquet mauris. Aliquam ornare odio sit amet urna placerat vestibulum. Aenean a cursus mauris, quis vulputate erat. Nullam convallis scelerisque ligula, at finibus lectus laoreet at. 

\noindent Keywords:
%
%-----------Ou digite aqui o seu resumo em Frances----------
%\resumo{Résumé} 
%\noindent Mots-clés: 

%-----------Sumário, lista de figura e de tabela------------
\listoffigures \thispagestyle{empty}
%\listoftables \thispagestyle{empty}
\tableofcontents

%-----------Início do Conteúdo------------------------------
\pagestyle{ruledheader}

%%%%%%%%%%%%%%%%%%%%%%%%%%%%%%%%%%%%%%%%%%%%%%%%%%%%%%%%%%%%%%%%%%%%%%%%%%%%%%%%
%
% Hifenização - Colocar lista de palavras que não devem ser separadas ou que não estão no dicionario português.
% As palavras do dicionario português já são separadas corretamente pelo LateX
%
\hyphenation{ Hardware Software }

%%%%%%%%%%%%%%%%%%%%%%%%%%%%%%%%%%%%%%%%%%%%%%%%%%%%%%%%%%%%%%%%%%%%%%%%%%%%%%%%
%% A partir daqui coloque seus capítulos. Sugere-se que eles sejam inseridos com o comando \input
%% Da seguinte maneira:

\chapter{INTRODUÇÃO}
\label{cap:introducao}

 Lorem ipsum dolor sit amet, consectetur adipiscing elit. Sed imperdiet lacus consectetur vestibulum scelerisque. Integer accumsan odio nisi, sed aliquet quam consequat tincidunt. Ut at sollicitudin felis. Duis tempor condimentum velit ac molestie. Donec vitae luctus velit, vitae faucibus mi. Suspendisse potenti. Mauris accumsan mi quis neque aliquet ultricies.

Suspendisse porta ultricies turpis, id porta risus. Sed nec bibendum ligula. Praesent sapien tortor, condimentum ut interdum quis, ornare ac nisi. Phasellus blandit ipsum ac mollis porta. Nunc egestas elementum est, sit amet hendrerit leo finibus placerat. Etiam finibus lorem quis dolor pellentesque, eu hendrerit nisl rhoncus. Aenean a mi consectetur, efficitur justo non, aliquam risus. Sed feugiat nisl vitae venenatis pulvinar. Suspendisse molestie quis tellus eget fringilla. Nunc at posuere tortor. Sed venenatis dui risus, non congue magna vehicula eu.

Morbi efficitur molestie pellentesque. Fusce tincidunt vitae dolor ac ornare. Mauris nibh mi, condimentum nec ex a, semper posuere augue. Ut sagittis condimentum lacus, et lacinia sem ornare nec. Praesent cursus sagittis lacus ut iaculis. Nunc faucibus, elit ac imperdiet malesuada, velit est faucibus diam, vitae ullamcorper sapien augue a nisi. Morbi consectetur pulvinar felis vel feugiat. Phasellus tempor magna eget purus placerat luctus. Vestibulum bibendum dapibus arcu in semper. Phasellus vel porta mauris. Ut nec mauris vel ante auctor vulputate a sed nisi. Nam ullamcorper purus vel dolor interdum efficitur. Aenean rhoncus mollis porta. Vivamus est urna, finibus vel leo at, porta tempus sem. 

\section{Justificativa}


\section{Objetivos}
\subsection{Gerais}

\subsection{Específicos}
\begin{itemize}
    \item Item 1
    \item Item 2
    \item Item 3 
\end{itemize}

 				    % Capítulo Introdução
%\input{cap_revisao_literatura}             % Capítulo Revisão de Literatura
\chapter{ESTADO DA ARTE}
\label{cap:estadoarte}
		            % Capítulo Estado da arte
\chapter{MATERIAIS E MÉTODOS}
\label{cap:projeto}
\section{Materiais}

\section{Métodos}
			    % Capítulo Desenvolvimento
\chapter{RESULTADOS}
\label{cap:resultados}

                    % Capítulo Resultados e Discussões
\chapter{CONCLUSÕES}
\label{cap:conclusoes}

\section{Trabalhos Futuros}

				    % Capítulo Conclusões

%------------------------------Bibliografia-------------------------------------

%\bibliographystyle{abnt-alf}
\printbibliography % arquivos com as entradas bib.

%------------------------------Apêndices----------------------------------------
\appendix
\chapter{APÊNDICE}
\label{cap:apendice}
Para adicionar outros apêndices ou anexos basta usar adicionar capítulos a este arquivo. 
\section{Apêndice 1}
\begin{enumerate}
    \item Item 1
\end{enumerate}
					

\end{document}
